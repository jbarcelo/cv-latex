% LaTeX resume using res.cls
\documentclass[line,margin]{res} 

\usepackage[utf8]{inputenc}

%\usepackage{epsfig}
%\usepackage{helvetica} % uses helvetica postscript font (download helvetica.sty)
%\usepackage{newcent}   % uses new century schoolbook postscript font 

\begin{document}
{\huge Contact Information}

\section{\textsc{Given name}}
Jaume

\section{\textsc{Surnames}}
Barcelo Vicens

%\section{\textsc{Nationality}}
%Spanish
%
%%\section{\textsc{Passport No.}}
%%43135949
%
\section{\textsc{E-mail addresses}}
jaume.barcelo@uc3m.es \\
jaume.barcelo@gmail.com (personal)

\section{\textsc{Mobile phone number}}
(+34) 686 753020 \\

\section{\textsc{Mailing addresses}}
%
(Office address) \\
Universidad Carlos III de Madrid.\\
Avda. de la Universidad, 30\\
28911 Leganes, Madrid, Spain\\
 \\
(Home address) \\
Residencia Fernando de los Rios.\\
Avda. de las Ciudades, 1\\
28903 Getafe, Madrid, Spain\\

\section{\textsc{Web address}}
http://www.jaumebarcelo.info

\section{\textsc{References}}
For references, please contact:

%Arturo Azcorra\\
%Universidad Carlos III de Madrid, professor.\\
%IMDEA Networks, director. \\
%CDTI, director.\\
%email: azcorra@it.uc3m.es

Albert Banchs\\
Universidad Carlos III de Madrid, associate professor.\\
IMDEA Networks, deputy director.\\
email: banchs@it.uc3m.es

Miquel Oliver \\
Universitat Pompeu Fabra, NeTS research group coordinator.\\
email: miquel.oliver@upf.edu

%Alberto Lopez-Toledo \\
%Telefonica Research Labs, senior researcher.\\
%email: alopezt@tid.es

%Guillem Femenias \\
%Universitat de les Illes Balears, professor.\\
%email: guillem.femenias@uib.es\\


%\newpage

%{\huge Cover Letter}
%\medskip\hrule height 1pt
%\begin{flushright}
%\hfill Jaume Barcelo\\
%\hfill Residencia Fernando de los Rios.\\
%\hfill Avda. de las Ciudades, 1\\
%\hfill 28903 Getafe, Madrid\\
%2 de octubre de 2010
%\end{flushright} 
%\begin{flushleft}
%{\large\bf Jaume Barcelo}

%\vspace{8cm} 

%2 de octubre de 2010

%Prof. Dr. Carlos Elías Pérez\\
%Director\\
%Residencia Fernando de los Ríos\\
%Avda. de las Ciudades, 1\\
%28903 Getafe, Madrid\\

%\end{flushleft}
%\vfill % forces letterhead to top of page

 
%\opening{Estimado Prof. Dr. Carlos Elías Pérez:} 

%\noindent Me dirijo a usted tras recibir de la administración del Colegio Mayor información sobre una beca para apoyo a la dirección. Quisiera comunicarle mi disponibilidad para colaborar en las tareas que usted considere oportunas para la gestión y el buen funcionamiento de la Residencia.


%Atentamente,

%\vspace{2cm} 
%\epsfig{file=signature.eps, height=0.7in}
%

%Jaume Barcelo
 

\newpage
%%{\huge Financial Plan}
%%
%%
%%\vspace{2cm}
%%
%%Salary: $\sim$ 36.000 Eur
%%
%%Travel and conference budget: $\sim$ 15.000 Eur
%%
%%Hardware and testbed: $\sim$ 7.000 Eur
%%
%%Other expenses: $\sim$ 2.000 Eur
%%
%%\vspace{0.5cm}
%%
%%Total research costs:  $\sim$ 60.000 Eur
%%
%%\newpage
%%{\huge Research Plan}
%%
%%Carrier Sense Multiple Access with Enhanced Collision Avoidance (CSMA/ECA) is a novel Medium Access Control (MAC) mechanism that substantially reduces the chances of collisions in Wireless Local Area Networks (WLANs). Even though there are other proposals for new MAC mechanism for WLANs, the uniqueness of CSMA/ECA comes from two key advantages: backward compatibility with currently implemented protocols and simplicity.
%%
%%So far, CSMA/ECA has been validated by two different simulators (one implemented using octave and the other implemented using the Component Oriented Simulation Toolkit, COST) by two different researchers. The performance of CSMA/ECA has also been analytically modelled, obtaining results in agreement with the simulations.
%%
%%However, there are still many pending tasks related to CSMA/ECA. First, its performance and benefits need to be assessed by independent international teams to guarantee the objectivity of the results. It is also necessary to consider all the possible scenarios (such as challenging channel conditions) to present evidence that CSMA/ECA always outperforms CSMA/CA.
%%
%%In order to disseminate the idea and foster the research on CSMA/ECA, a strong publication effort is needed, targeting top conferences and journals related to the field of wireless communications. It is also necessary to prepare a module for a well known network simulator (such as NS-3) to facilitate the task of all the researchers willing to study CSMA/ECA.
%%
%%However, there are situations and phenomena that would never be detected in a simulation environment. Thus, it is necessary to go one step beyond and prepare a testbed to perform experimental research. However, the lack of WLAN prototypes for research purposes hinders advances in this direction. It is likely that the collaboration of manufacturers is required to modify the firmware of commercial WLAN cards to implement CSMA/ECA.
%%
%%Even though the majority of current WLANs involve infrastructure and one-hop communication, this situation might change in the upcoming years. Thanks to the new standard amendment IEEE 802.11s, WLANs would support multi-hop communication. It is a research challenge to explore both analytically and by simulations the performance of CSMA/ECA in IEEE 802.11s networks. Our belief is that fixed slot size is a requirement for collision-free operation in multi-hop networks. Nevertheless, more evidence is needed to support this conjecture.
%%
%%\newpage
%%{\huge Effort Plan}
%%
%%\vspace{2cm}
%%
%%\begin{itemize}
%%  \item Research $\sim$ 60\%
%%  \begin{itemize}
%%    \item NS-3 module for CSMA/ECA $\sim$ 3 person month.
%%    \item CSMA/ECA testbed and demonstrator $\sim$ 6 person month.
%%    \item Extension of CSMA/ECA to multi-hop networks $\sim$ 6 person month.
%%    \item Feasibility study for CSMA/ECA for technologies different than IEEE 802.11 $\sim$ 3 person month.
%%  \end{itemize}
%%  \item Teaching and student guidance $\sim$ 20\%
%%  \item Services $\sim$ 10\%
%%  \begin{itemize}
%%    \item Reviews
%%    \item Events organization
%%  \end{itemize}
%%  \item Other $\sim$ 10\%
%%  \begin{itemize}
%%    \item Dissemination
%%    \item Technology transfer
%%  \end{itemize}
%%\end{itemize}
%%
%%\newpage
%%
%%
%%\name{Susan R. Bumpershoot}
%% \address used twice to have two lines of address
%%\address{1985  Storm Lane, Troy, NY 12180}
%%\address{(518) 273-0014 or (518) 272-6666}
%
% 
\begin{resume}
{\huge Research CV and Teaching Experience}

\section{Short Bio}
      Jaume Barcelo is a postdoctoral researcher with Universidad Carlos III de Madrid. His research interests are in the field of wireless communications in general and in MAC protocols for WLANs in particular. Jaume holds a Ph.D. from Universitat Pompeu Fabra, where he worked under the supervision of Prof. Miquel Oliver. He obtained his M.Eng. from Universitat Politecnica de Catalunya and his B.Eng. from Universitat de les Illes Balears. Jaume also worked as a research assistant at the Kaiserslautern Universitaet in the group leaded by Prof. Paul W. Baier.

\section{\textsc{Education}} 
\textbf{Universitat Pompeu Fabra} \hfill 2009 \\
Ph.D. in Computer Science and Digital Communications \\
 Thesis: Carrier Sense Multiple Access with Enhanced Collision Avoidance. \\
 Advisor: Prof. Miquel Oliver\\
\textbf{Universitat Politecnica de Catalunya} \hfill 2004 \\ 
Master of Science in Telecommunications \\
 Thesis: Deployment and configuration of an Open Access Network at the UPF. \\
 Advisor: Prof. Miquel Oliver.\\
\textbf{Universitat de les Illes Balears/Kaiserslautern Universitaet} \hfill 2001 \\ 
Bachelor of Science in Computer Communications \\
Thesis: Impact of time variant channels on the performance of Joint Transmission downlinks.\\
Advisor: Dr. Hendrik Troeger. / Dr. Ignasi Furio \\

\section{\textsc{Best Paper Awards}}

\hfill {MACOM'11} \\
J. Barcelo, B. Bellalta,  M. Oliver, A. Banchs, \\
\textbf{"Collision-Free Operation in Ad-Hoc Networks"}\\

\hfill {MACOM'10} \\
J. Barcelo, B. Bellalta,  C. Cano, A.Sfairopoulou, M. Oliver,\\
\textbf{"Dynamic Parameter Adjustment in CSMA/ECA"}\\

\hfill {ISWCS'09} \\
C. Cano, B. Bellalta, J. Barcelo, M. Oliver, A. Sfairopoulou\\
\textbf{"Analytical Model of the LPL with Wake up after Transmissions MAC protocol for WSNs"}\\

\hfill {WWIC'09} \\
C. Cano, B. Bellalta, J. Barcelo, A. Sfairopoulou\\
\textbf{"A Novel MAC Protocol for Event-based Wireless Sensor Networks: Improving the Collective QoS"}\\

\hfill {Eunice'08 } \\
J. Barcelo, B. Bellalta,  C. Cano, M.Oliver,\\
\textbf{"Learning-BEB: Avoiding Collisions in WLAN"}\\

\section{\textsc{Invited Paper}}

\hfill {IEEE VTC Spring'09} \\
J. Barcelo, B. Bellalta,  C. Cano, A.Sfairopoulou, M. Oliver,\\
\textbf{"CSMA with Enhanced Collision Avoidance: a Performance Assessment"}\\

\section{\textsc{Journals}}

\hfill {IEEE Comm. Letters} \\
J. Barcelo, H. Inaltekin, B. Bellalta,\\
\textbf{"Obey or Play: Asymptotic Equivalence of Slotted Aloha with a Game Theoretic Contention Model"}\\

\hfill {EURASIP Journal on Wireless Communications and Networking} \\
J. Barcelo, B. Bellalta, C. Cano, A. Sfairopoulou, M. Oliver, K. Verma,\\
\textbf{"Towards a Collision-Free WLAN: Dynamic Parameter Adjustment in CSMA / E2CA"}\\

\hfill {Springer Wireless Personal Communications 2011} \\
B. Bellalta, J. Barcelo, M. Oliver,\\
\textbf{"The role of the queueing process in the performance of Downlink SDMA systems"}\\

\hfill {Elsevier Mobile Networks and Applications 2011} \\
C. Cano, B. Bellalta, A. Sfairopoulou, J. Barcelo,\\
\textbf{"Taking Advantage of Overhearing in Low Power Listening WSNs: A Performance Analysis of the LWT-MAC Protocol"}\\

\hfill {Elsevier Computer Networks 2010} \\
C. Cano, B. Bellalta, A. Sfairopoulou, J. Barcelo,\\
\textbf{"Tuning the EDCA Parameters in WLANs with Heterogeneous Traffic: A Flow-level Analysis"}\\

\hfill {IEEE Comm. Letters 2009} \\
C. Cano, B. Bellalta, A. Sfairopoulou, J. Barcelo\\
\textbf{"A Low Power Listening MAC with Scheduled Wake Up after Transmissions for WSNs"}\\

\hfill {ACM Mobile Computing and Communications Review 2008} \\
J. Barcelo, A. Sfairopoulou, B. Bellalta,\\
\textbf{"Wireless Open Metropolitan Access Networks"}\\

\section{\textsc{Workshops and conferences}}

\hfill {IEEE ICC'10} \\
J. Barcelo, A. Lopez-Toledo,  C. Cano, M. Oliver,\\
\textbf{"Fairness and convergence of CSMA with Enhanced Collision Avoidance (ECA)"}\\

\hfill {NEUTRAL'09} \\
J. Barcelo,  J. Infante, M. Oliver,\\
\textbf{"Wireless Open Access Networks: State-of-the-Art and Technological Opportunities "}\\

\hfill {IWCMC'09} \\
J. Barcelo, B. Bellalta,  C. Cano, A.Sfairopoulou, M. Oliver,\\
\textbf{"Carrier Sense Multiple Access with Enhanced Collision Avoidance: a Performance Analysis"}\\

\hfill {MACOM (IEEE ICC'09)} \\
J. Barcelo, B. Bellalta,  C. Cano, A.Sfairopoulou, M. Oliver, J. Zuidweg,\\
\textbf{"Traffic Prioritization for Carrier Sense Multiple Access with Enhanced Collision Avoidance"}\\

\hfill {Jitel'08} \\
J. Barcelo, B. Bellalta, A. Sfairopoulou, C. Cano, M. Oliver,\\
\textbf{"Dynamic P-Persistent Backoff for Higher Efficiency and Implicit Prioritization"}\\

\hfill {IEEE/IFIP WONS'08} \\
J. Barcelo, B. Bellalta,  C. Cano, A.Sfairopoulou,\\
\textbf{"VoIP Packet Delay in Single-Hop Ad-Hoc IEEE 802.11 Networks"}\\

\hfill {in book IFIP Vol. 265} \\
J. Barcelo, B. Bellalta,  C. Cano, A.Sfairopoulou,\\
\textbf{"No Ack in IEEE 802.11e Single-Hop Ad-Hoc VoIP Networks"}\\

\hfill {WEBIST'07} \\
J. Barcelo, B. Bellalta, C.Macian, M.Oliver, A.Sfairopoulou,\\
\textbf{"Position Information for VoIP Emergency Calls Originating from a Wireless Metropolitan Access Network"}\\

\hfill {IEEE MCWC'06} \\
J. Barcelo, C.Macian, P.Novell, M.Isidre, E. Arago,\\
\textbf{"Offering VoIP Services in a Wireless Neutral Operator Environment"}\\

\hfill {ICQT (ACM Sigmetrics'06)} \\
J. Barcelo, M.Oliver, J.Infante,\\
\textbf{"Adapting a Captive Portal to enable SMS-based Micropayment for Wireless Internet Access"}\\

\hfill {IEEE TRIDENTCOM'06} \\
J. Barcelo, C.Macian, J.Infante, M.Oliver, A.Sfairopoulou,\\
\textbf{"Barcelona's Open Access Network Testbed"}\\

\newpage
\section{\textsc{Research Grant}}

\hfill {A4U 2010-2011} \\
\textbf{"Postdoctoral award. Media Access Control and Wireless Mesh Networks."}\\

\section{\textsc{Research Projects}}

\hfill {CICYT 2009-2011} \\
Project Coordinator: Miquel Oliver\\
\textbf{"Gestión P2P de tráfico multimedia con calidad de servicion sobre redes malladas: Gepeto"}\\

\hfill {COST290 2004-2008} \\
Project Coordinator: Y. Koucheryavy\\
\textbf{"Traffic and QoS Management in Wireless Multimedia Networks"}\\

\hfill {FP6 2003-2007} \\
Project Coordinator: Sergio Benedetto\\
\textbf{"Network of Excellence in Wireless Communications"}\\

\hfill {CICYT 2003-2006} \\
Project Coordinator: Joan Vinyes / Miquel Oliver\\
\textbf{"Servicios Inteligentes Móviles Avanzados"}\\

\section{\textsc{Teaching Experience}}
Jaume Barcelo is currently coordinating complex courses with multiple groups, different languages and up to five different teachers.
He is also teaching both at the graduate and undergraduate level at Universidad Carlos III de Madrid.
One of the courses he has taught was included in the joint UC3M-UPC master program, with students from UC3M, UPC, IMDEA networks and CTTC.
The students in Barcelona remotely followed the course using teleconference.
Jaume has also taught for four years at Universitat Pompeu Fabra.
He has covered different topics in the wide area of telecommunications, ranging from cryptography to network application and transport layer and prototyping of services for mobile devices.
His experience includes lectures, seminars and laboratories.

\section{\textsc{Courses Taught}}
Lecturer
\hfill {Fall'10, Fall'11}\\
Cryptography and Network Security (Grad. School) \\
\newline
Lecturer
\hfill {Fall'10, Fall'11}\\
Criptografía y seguridad en redes\\
\newline
Lecturer
\hfill {Spring'10, Spring'11}\\
WLAN Seminar (Grad. School)\\
\newline
Lecturer
\hfill {Spring'10, Spring'11}\\
Laboratorio de Ingeniería de Servicios\\
\newline
Lecturer
\hfill {Fall'08}\\
Xarxes i Serveis I\\
\newline
Lecturer
\hfill {Spring'08}\\
Laboratori de Comunicacions\\
\newline
Teaching Assistant
\hfill {Fall'07, Spring'07, Spring'06, Spring'05}\\
Xarxes i Serveis, Laboratori de Telemàtica III, Enginyeria de sistemes de transmissió

\section{\textsc{Services}}
Organizer
\hfill {Eunice'09 (Publicity)}

TPC
\hfill {Neutral'09, MACOM'10, MACOM'11, MESHTECH'11}

%Reviewer
%\hfill {ISTAS'10, VTC'09, IEEE TMM, JITEL'08, EUNICE'08, IJCS, ACCESSNETS'07.}\\
%

\section{\textsc{Internships}}

\employer{\textbf{HP}}
\title{Solutions and Customization Labs Intern}
\location{Sant Cugat, Catalunya, Spain}
\dates{2003-2004}
\begin{position}
Developed internal tools to assist the validation of the customized products.
\end{position}

\employer{\textbf{ID Tech Camps}}
\title{Camp leader and instructor}
\location{Stanford, Palo Alto, CA}
\dates{Summer 2002}
\begin{position}
Web Design, Multimedia and Game Creation, Programming and Robotics.
\end{position}

\employer{\textbf{Air Europa Airlines}}
\title{Electronic Sales Intern}
\location{Lluchmajor, Mallorca, Spain}
\dates{Summer 2000}
\begin{position}
On-line ticket sales developer and maintainer (Html/Wml/Javascript, PL/SQL, Oracle Application Server).
\end{position}


\end{resume}
\end{document}



